\section{Results}

Our experts have reported a variety of problems from different aspects of our interface. We will categorise their findings regarding their severity into two categories, minor and major.(cite the paper maybe) This will be used to better prioritise them. 

\subsection{Major severity problems}

In this section the problems that were quantified as "Major" by at least one evaluator are listed. \\

\noindent
\textbf{Problem 1.} The card edit menu does not have a cancel button. Therefore, users are not able to reset the card to the content before start of the edit. The ability to restore the state of the card is highly desireable. A cancel button can be added that allows to restore the initial state of the card, discarding any new changes.

\noindent
\textbf{Problem 2.} Tag edit menu does not have a cancel button. Hence, users are not able to exit from the menu of creation of the new tag, thus not limitating the freedom of users. Moreover, since the period to recall information stored outside of the working memory is larger than reaction time, it is quite likely that uses will enter the menu to create new tag automatically, and only after a small delay remember that there already exists the desired tag. The "Cancel" button should be added.

\noindent
\textbf{Problem 3.} It is not clear what does task colour mean in customization menu, which forces users to spend additional thought on what it could be. While, after a little time it becomes clear that it references the background color of the cards, however this addition strain is undesirable. Labels content should be changed to the "Card color".

\subsection{Minor severity problems}

In this section the problems that were quantified as "Minor" by all evaluators are listed. \\

\noindent
\textbf{Problem 4.} The task colour in the customization menu does not have a reset button. There might be cases when users don't remember the exact color that was set for the task before, and therefore such a button should be added.

\noindent
\textbf{Problem 5.} There is no separation between board names on the sidebar. This leads to possible miss clicks on the boards, and thus might require additional time and stress for users to figure out that they are on the undesirable board. There are many ways to separate different boards on the sidebar, however, the way we prefer is adding a border inbetween board names.

\noindent
\textbf{Problem 6.} The fields in the add board menu are in the not convenient order. The first label average user reads on the screen is the top one, which is "Add new board by ID". However, the second label "Create new board" we expect to be significantly more frequent choice. Thus, to minimize expected time spend by the user on this menu, the two labels and corresponding text fields should be swapped.

\noindent
\textbf{Problem 7.} There is no indication of how the board ID can be closed after sharing. While the cases when the ID will dissapear are noted alongside prototype, there is no way for user to determine these ways just by looking at the overview. The solution we propose is to change the "Share board" button to "Hide ID" button while the ID is shown, and, moreover, hide it automatically 30 seconds after it was shown.

\noindent
\textbf{Problem 8.} The boards can be deleted to easily. Accidental click on delete board button leads to instantenous deletion of the board, which is undesired behaviour. While indeed it is clearly indicated on the button, the action of board removal cannot be reset, thus there must be a confirmation prompt.


Some table here


\subsection{Conclusions and Improvements}


\begin{itemize}

    \item{In the card editing menu we added a cancel button so that the user can discard the changes he made and restore the previous state of the card.}

    \item{In a customization menu, for a task colour field we added a reset button so that the user can revert the colour changes he made in one click, instead of manually picking the same colour as it was before.}

    \item{In a tag editing menu we added a cancel button which discards the changes made to the tags so users can go back to the previous organisation of tags.}

    \item{We decided to surround the names of the boards with borders so that it is clear to distinguish and click on them.}

    \item{We made the decision to change the order of the “join” and “create” buttons.}

    \item{When the user clicks the “SHARE BOARD” button, the text of that button changes to the “CANCEL SHARE” so that it is clear how to get back to the default view of the board and clear the line with the board ID.}

    \item{The text was changed from “Task colour” to “Card colour”, therefore the naming convention of individual tasks is consistent throughout the whole application.}

    \item{We added a menu that pops up everytime the user tries to delete a board and asks them if they are sure that they want to delete the board, together with all its contents.}

\end{itemize}
